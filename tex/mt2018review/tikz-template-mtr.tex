\documentclass{article}
\usepackage[T1]{fontenc}
\usepackage[utf8]{inputenc}
\usepackage{pgfplots}
\pgfplotsset{compat=newest}
\usepackage{pgfplotstable}
\usetikzlibrary{plotmarks}
\usepackage{amsmath}
\usepackage{amsfonts}
\usepackage{pdflscape}
\usepackage{fullpage}
\DeclareMathAlphabet\mathbb{U}{fplmbb}{m}{n} %Preferred style of \mathbb
\DeclareMathOperator{\Exp}{\mathbb{E}}

\thispagestyle{empty} % No page number so you can crop correctly

%%%%%%%%%%%%%%%%%%%%%%%%%%%%%%%%%%%%%%%%%%%%%%%%%%%%%%%%%%%%%%%%%%%%%%%%%%%%%%%%
% MUSTACHE VARIABLES
%
% The variables above are used below as placeholders.
% They are fenced by "<<" and ">>" instead of the typical "{{" and "}}" because
% braces carry meaning in LaTeX.
% These variables are also the keys in a dictionary and the values in this
% dictionary will replace the corresponding placeholders in this template.
%
% m0segments: m_0(u)
% m1segments: m_1(u)
% mtesegments: difference between m_1(u) and m_0(u)
%   Notes:
%     Each of these variables is a vector of dictionaries.
%     Each dictionary has a single key called "coordinates" that contains the
%     segment-specific coordinates.
%
%%%%%%%%%%%%%%%%%%%%%%%%%%%%%%%%%%%%%%%%%%%%%%%%%%%%%%%%%%%%%%%%%%%%%%%%%%%%%%%%

\begin{document}
\begin{landscape}
\begin{center}

\begin{tikzpicture}

\begin{axis}[
    width=6in,
    height=4in,
    ymin=-1,
    ymax=1,
    xmin=0,
    xmax=1,
    major tick length=.075cm,
    xlabel={Unobserved Heterogeneity in Treatment Choice ($u$)},
    ylabel={Parameter Value},
    legend cell align=left,
    mark repeat = {10},
    %cycle list name=black white,
    cycle list name=exotic,
]

<<#:m0segments>>
\addplot
coordinates{
  <<coordinates>>
};
<</:m0segments>>

<<#:m1segments>>
\addplot
coordinates{
  <<coordinates>>
};
<</:m1segments>>

<<#:mtesegments>>
\addplot
coordinates{
  <<coordinates>>
};
<</:mtesegments>>

\legend{$m_{0}(u)$, $m_{1}(u)$, $m_{1}(u) - m_{0}(u)$}

\end{axis}

\end{tikzpicture}

\end{center}
\end{landscape}
\end{document}
