\documentclass{article}
\usepackage[T1]{fontenc}
\usepackage[utf8]{inputenc}
\usepackage{pgfplots}
\pgfplotsset{compat=newest}
\usepackage{pgfplotstable}
\usetikzlibrary{plotmarks}
\usepackage{amsmath}
\usepackage{amsfonts}
\usepackage{pdflscape}
\usepackage{fullpage}
\DeclareMathAlphabet\mathbb{U}{fplmbb}{m}{n} %Preferred style of \mathbb
\DeclareMathOperator{\Exp}{\mathbb{E}}

\thispagestyle{empty} % No page number so you can crop correctly

%%%%%%%%%%%%%%%%%%%%%%%%%%%%%%%%%%%%%%%%%%%%%%%%%%%%%%%%%%%%%%%%%%%%%%%%%%%%%%%%
% MUSTACHE VARIABLES
%
% The variables above are used below as placeholders.
% They are fenced by "<<" and ">>" instead of the typical "{{" and "}}" because
% braces carry meaning in LaTeX.
% These variables are also the keys in a dictionary and the values in this
% dictionary will replace the corresponding placeholders in this template.
%
% ypos
% ylabel
%
% curves -- vector of dictionaries whose keys are:
%   opts: string following `\addplot`
%   coordinates: string of (x, y) pairs
%   shift: optional, shifts the `cycle list`
%
%%%%%%%%%%%%%%%%%%%%%%%%%%%%%%%%%%%%%%%%%%%%%%%%%%%%%%%%%%%%%%%%%%%%%%%%%%%%%%%%

\begin{document}
\begin{landscape}
\begin{center}

\begin{tikzpicture}

\begin{axis}[
    width=6in,
    height=4in,
    major tick length=.075cm,
    enlarge x limits=0,
    ymin = -1,
    ymax = .3,
    xtick=data,
    xlabel={Polynomial Degree ($K_{0} = K_{1} \equiv K$)},
    ylabel={Upper and Lower Bounds},
    legend cell align=left,
    legend style = {at={(0,0)}, anchor = south west},
    cycle list name=exotic,
    tick pos = left,
    extra y ticks={<<:ypos>>},
    extra y tick labels={<<:ylabel>>},
    extra y tick style={
        tick pos=right,
        ticklabel pos=right,
        align=left,
    },
]

<<#:curves>>
\addplot<<opts>>
coordinates{
    <<coordinates>>
};

<<#shift>>
\pgfplotsset{cycle list shift=<<shift>>}
<</shift>>
<</:curves>>

\legend{
    Polynomial,
    Nonparametric,
    Polynomial and decreasing,
    Nonparametric and decreasing,
}

\end{axis}

\end{tikzpicture}

\end{center}
\end{landscape}
\end{document}
