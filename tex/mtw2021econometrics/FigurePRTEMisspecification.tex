\documentclass{standalone}
\usepackage{pgfplots}
\pgfplotsset{compat=newest}
\usepackage{pgfplotstable}
\usetikzlibrary{arrows}
\usetikzlibrary{plotmarks}

\usepackage{amsmath}
\usepackage{amsfonts}
\usepackage{bm}
\DeclareMathAlphabet\mathbb{U}{fplmbb}{m}{n} %Preferred style of \mathbb
\DeclareMathOperator{\Exp}{\mathbb{E}}

%%%%%%%%%%%%%%%%%%%%%%%%%%%%%%%%%%%%%%%%%%%%%%%%%%%%%%%%%%%%%%%%%%%%%%%%%%%%%%%%
%%%%%%%%%%%%%%%%%%%%%%%%%%%%%%%%%%%%%%%%%%%%%%%%%%%%%%%%%%%%%%%%%%%%%%%%%%%%%%%%
%%%%%%%%%%%%%%%%%%%%%%%%%%%%%%%%%%%%%%%%%%%%%%%%%%%%%%%%%%%%%%%%%%%%%%%%%%%%%%%%
\pgfplotstableread[col sep = comma]{prte-misspecification.csv}\data
\pgfplotstablegetrowsof{\data} % get number of rows of data
\pgfmathsetmacro{\rows}{\pgfplotsretval-1} % 0-indexed => shift number of rows by 1

% DRAW A RECTANGLE
% param: left boundary
% param: right boundary
% param: vertical location
% param: 0.5 * height
\newcommand\drawrect[4]{
    \edef\tmp{\noexpand\draw [fill=black] (#1, #3-#4) rectangle (#2, #3+#4);}
    \tmp
}

% ROUND NUMBER TO SOME LEVEL OF PRECISION
% param: number to be rounded
% param: level of precision
% REF: https://tex.stackexchange.com/questions/183852/how-can-i-round-numbers-relative-to-their-order-of-magnitude
\newcommand{\mynum}[2]{
  \pgfkeys{/pgf/fpu}
  \pgfmathparse{#1}
  \pgfmathfloattofixed\pgfmathresult
  \pgfmathprintnumberto[
    fixed relative,
    precision=#2,
    verbatim,
  ]{\pgfmathresult}{\tmp}%
  \tmp
}

\tikzset{
    vline/.style={dotted, thick}, % vertical line style
}

% MACROS
\def\xmin{-0.17}
\def\xmax{0.22}
\def\ymin{0}
\def\ymax{5}

\begin{document}
\thispagestyle{empty} % Get rid of page number

% TRUE VALUE COORDINATES
\pgfplotstablegetelem{0}{[index]1}\of{\data}
\edef\tvlb{\pgfplotsretval} % \tvlb: lower bound of true value
\pgfplotstablegetelem{0}{[index]2}\of{\data}
\edef\tvub{\pgfplotsretval} % \tvub: upper bound of true value
% NOTE: \tvlb and \tvub should be the same.

\begin{tikzpicture}
    \begin{axis}[
        title={Bounds on a PRTE},
        % X-AXIS
        x post scale=2,
        xlabel={PRTE},
        axis x line=center,
        xlabel style={anchor=north, xshift=3mm, yshift=-0.5mm},
        xticklabel style={
            /pgf/number format/fixed,
            /pgf/number format/precision=2
        }, % remove scientific notation of labels (http://tex.stackexchange.com/questions/119887/ddg#119888)
        extra x ticks={\tvlb},
        extra x tick labels = {\mynum{\tvlb}{3}},
        axis x line=bottom,
        xmin=\xmin,
        xmax=\xmax,
        xmajorgrids = true,
        xtick style={draw=none},
        x axis line style = {very thick, ->, >=stealth'},
        % Y-AXIS
        ylabel={Selection assumptions}, % Standard case
        axis y line=center, % somehow puts y-axis label on top...
        ylabel style={anchor=east, text width=2.9cm, align=center},
        yticklabels ={
            ,
            ,
            Both instruments with IAM,
            Instrument~1 with IAM/PM,
            Instrument~2 with IAM/PM,
            Both instruments with PM and mutual consistency
        },
        yticklabel style={text width=2.9cm, align=center}, % modify text width to control wrapping of y-axis tick labels
        ytick style={draw=none},
        axis y line=left, % else: both left and right
        ymin=\ymin,
        ymax=\ymax,
        y dir=reverse, % by default, first row of CSV starts at bottom of y-axis; we want it the other way around
        y axis line style = {very thick, <-, >=stealth'}, % y dir = reverse so I need to flip the arrow
        label style={font=\bfseries},
        title style={font=\bfseries},
    ]

    % TRUE VALUE VERTICAL LINE
    \draw[vline] (\tvlb, 0) -- (\tvub, 5);
    % TRUE VALUE LABEL
    \draw[very thick, <-, >=stealth'] (\tvlb, 0 + 0.6) -- (\tvlb + 0.02, 0 + 0.4) node[anchor=west,fill=white] {True value};

    % x = 0 VERTICAL LINE
    \draw[vline] (0, 0) -- (0, 5);

    % BARS
    % REF: https://tex.stackexchange.com/questions/170664/foreach-not-behaving-in-axis-environment
    \pgfplotsinvokeforeach{1,...,\rows}{ % skip first row (i.e., skip row index 0)
        % #1 is the iterator of the for loop
        % Iterate from the second row (index 1) to the final row (\rows)
        \pgfplotstablegetelem{#1}{[index]1}\of{\data}
        \edef\lb{\pgfplotsretval} % save the value of #1-row and second column (index 1) in \lb
        \pgfplotstablegetelem{#1}{[index]2}\of{\data}
        \edef\ub{\pgfplotsretval} % save the value of #1-row and third column (index 2) in \ub
        \drawrect{\lb}{\ub}{#1}{0.25} % Draw rectangle of height 0.25 * 2 = 0.5
    }
    \end{axis}
\end{tikzpicture}

%%%%%%%%%%%%%%%%%%%%%%%%%%%%%%%%%%%%%%%%
\end{document}
