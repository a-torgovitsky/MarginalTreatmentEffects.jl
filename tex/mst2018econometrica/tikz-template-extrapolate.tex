\documentclass{article}
\usepackage[T1]{fontenc}
\usepackage[utf8]{inputenc}
\usepackage{pgfplots}
\pgfplotsset{compat=newest}
\usepackage{pgfplotstable}
\usetikzlibrary{plotmarks}
\usetikzlibrary{spy}
\usepackage{amsmath}
\usepackage{amsfonts}
\usepackage{pdflscape}
\usepackage{fullpage}
\DeclareMathAlphabet\mathbb{U}{fplmbb}{m}{n} %Preferred style of \mathbb
\DeclareMathOperator{\Exp}{\mathbb{E}}

\thispagestyle{empty} % No page number so you can crop correctly

%%%%%%%%%%%%%%%%%%%%%%%%%%%%%%%%%%%%%%%%%%%%%%%%%%%%%%%%%%%%%%%%%%%%%%%%%%%%%%%%
% MUSTACHE VARIABLES
%
% The variables above are used below as placeholders.
% They are fenced by "<<" and ">>" instead of the typical "{{" and "}}" because
% braces carry meaning in LaTeX.
% These variables are also the keys in a dictionary and the values in this
% dictionary will replace the corresponding placeholders in this template.
%
% lbsegments - lower bound
% ubsegments - upper bound
% truesegments - actual value
%   Notes:
%     Each of these variables is a vector of dictionaries.
%     Each dictionary has a single key called "coordinates" that contains the
%     segment-specific coordinates.
%
%%%%%%%%%%%%%%%%%%%%%%%%%%%%%%%%%%%%%%%%%%%%%%%%%%%%%%%%%%%%%%%%%%%%%%%%%%%%%%%%

\newcommand{\PLOTLINEWIDTH}{.9pt}

\begin{document}
\begin{landscape}
\begin{center}

\large{Bounds on $\text{LATE}(.35, \overline{u})$}

\begin{tikzpicture}[,
    font=\large,
    spy using overlays={rectangle, magnification=2.5, height=2cm, width=8cm, connect spies}
]

\begin{axis}[%
    width=6in,
    height=4in,
    ytick distance = .025,
    ymin=-.0499,
    ymax=.1499,
    yticklabel style={
        /pgf/number format/fixed,
        /pgf/number format/precision=3,
        /pgf/number format/skip 0.=true,
        font=\normalsize
    },
    scaled y ticks=false,
    xmin=.38,
    xmax=1,
    xtick distance=.05,
    xticklabel style={
        /pgf/number format/fixed,
        /pgf/number format/precision=2,
        /pgf/number format/skip 0.=true,
        font=\normalsize
    },
    xlabel={$\overline{u}$},
    axis lines* = left
]

\draw [dashed] (.9, -.05) -- (.9, .15);

% upper bound
<<#:ubsegments>>
\addplot[
    color=darkgray,
    solid,
    line width=\PLOTLINEWIDTH,
    mark = *,
    mark repeat = 5,
    mark size = .8pt
]
coordinates{
    <<coordinates>>
};
<</:ubsegments>>

% lower bound
<<#:lbsegments>>
\addplot[
    color=lightgray,
    solid,
    line width=\PLOTLINEWIDTH,
    mark = *,
    mark repeat = 5,
    mark size = .8pt
]
coordinates{
    <<coordinates>>
};
<</:lbsegments>>

% truth bound
<<#:truesegments>>
\addplot[
    color=black,
    solid,
    line width=\PLOTLINEWIDTH
]
coordinates{
    <<coordinates>>
};
<</:truesegments>>

\legend{Upper Bound, Lower Bound, Actual Value}

\end{axis}

\spy [gray] on (6, 5.3) in node [left] at (9,2);

\end{tikzpicture}

\end{center}
\end{landscape}
\end{document}
