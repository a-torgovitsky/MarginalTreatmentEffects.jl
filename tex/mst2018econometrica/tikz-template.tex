\documentclass{article}
\usepackage[T1]{fontenc}
\usepackage[utf8]{inputenc}
\usepackage{pgfplots}
\pgfplotsset{compat=newest}
\usepackage{pgfplotstable}
\usetikzlibrary{plotmarks}
\usetikzlibrary{calc}
\usepgfplotslibrary{fillbetween}
\usepackage{amsmath}
\usepackage{amsfonts}
\usepackage{ifthen}
\usepackage{pdflscape}
\usepackage{fullpage}
\usepgflibrary{patterns}
\usetikzlibrary{patterns}
\DeclareMathAlphabet\mathbb{U}{fplmbb}{m}{n} %Preferred style of \mathbb
\DeclareMathOperator{\Exp}{\mathbb{E}}

\thispagestyle{empty} % No page number so you can crop correctly
\pgfplotsset{set layers} % This is important --- shading won't work without it!

%%%%%%%%%%%%%%%%%%%%%%%%%%%%%%%%%%%%%%%%%%%%%%%%%%%%%%%%%%%%%%%%%%%%%%%%%%%%%%%%
% 05/20/16
% This allows you to put different line styles for different axes
% See
% http://tex.stackexchange.com/a/96910/1931
%%%%%%%%%%%%%%%%%%%%%%%%%%%%%%%%%%%%%%%%%%%%%%%%%%%%%%%%%%%%%%%%%%%%%%%%%%%%%%%%
\pgfplotsset{
    every first x axis line/.style={},
    every first y axis line/.style={},
    every first z axis line/.style={},
    every second x axis line/.style={},
    every second y axis line/.style={},
    every second z axis line/.style={},
    first x axis line style/.style={/pgfplots/every first x axis line/.append style={#1}},
    first y axis line style/.style={/pgfplots/every first y axis line/.append style={#1}},
    first z axis line style/.style={/pgfplots/every first z axis line/.append style={#1}},
    second x axis line style/.style={/pgfplots/every second x axis line/.append style={#1}},
    second y axis line style/.style={/pgfplots/every second y axis line/.append style={#1}},
    second z axis line style/.style={/pgfplots/every second z axis line/.append style={#1}}
}

\makeatletter
\def\pgfplots@drawaxis@outerlines@separate@onorientedsurf#1#2{%
    \if2\csname pgfplots@#1axislinesnum\endcsname
        % centered axis lines handled elsewhere.
    \else
    \scope[/pgfplots/every outer #1 axis line,
        #1discont,decoration={pre length=\csname #1disstart\endcsname, post length=\csname #1disend\endcsname}]
        \pgfplots@ifaxisline@B@onorientedsurf@should@be@drawn{0}{%
            \draw [/pgfplots/every first #1 axis line] decorate {
                \pgfextra
                % exchange roles of A <-> B axes:
                \pgfplotspointonorientedsurfaceabsetupfor{#2}{#1}{\pgfplotspointonorientedsurfaceN}%
                \pgfplots@drawgridlines@onorientedsurf@fromto{\csname pgfplots@#2min\endcsname}%
                \endpgfextra
                };
        }{}%
        \pgfplots@ifaxisline@B@onorientedsurf@should@be@drawn{1}{%
            \draw [/pgfplots/every second #1 axis line] decorate {
                \pgfextra
                % exchange roles of A <-> B axes:
                \pgfplotspointonorientedsurfaceabsetupfor{#2}{#1}{\pgfplotspointonorientedsurfaceN}%
                \pgfplots@drawgridlines@onorientedsurf@fromto{\csname pgfplots@#2max\endcsname}%
                \endpgfextra
                };
        }{}%
    \endscope
    \fi
}%
\makeatother

%%%%%%%%%%%%%%%%%%%%%%%%%%%%%%%%%%%%%%%%%%%%%%%%%%%%%%%%%%%%%%%%%%%%%%%%%%%%%%%%
% MUSTACHE VARIABLES
%
% The variables above are used below as placeholders.
% They are fenced by "<<" and ">>" instead of the typical "{{" and "}}" because
% braces carry meaning in LaTeX.
% These variables are also the keys in a dictionary and the values in this
% dictionary will replace the corresponding placeholders in this template.
%
% title
% titlesuffix
% titlevspace
% fontsize
% axiswidth
% axisheight
% xmin
% xmax
% markrepeat
% markphase
% mtrymin
% mtrymax
% xlabel
% weightymin
% weightymax
% ylabelweights
% ylabelbuffer
% ylabeltextwidth
% legendcols
% mtrlegendtext
% mtrylabel
%
% m0segments
% m1segments
%   Notes:
%     The MTR functions are split up into segments based on where
%     discontinuities arise.
%     Each segment will have its own dictionary with the following keys:
%       pathname
%       coordinates
%
% d0weights
% d1weights
%   Notes:
%     The weights are piecewise-constant functions of u.
%     Each piece will have its own dictionary with the following keys:
%       pathname
%       color
%       mark -- this isn't used in MT (2018)
%       marksize -- this isn't used in MT (2018)
%       coordinates
%       linetype -- this isn't used in MST (2018)
%     These dictionaries will be collected in two vectors,
%     one for d = 0 and another for d = 1.
%
% legend
%   Notes:
%     This variable is a vector of dictionaries.
%     Each dictionary contains the information of single entry in the legend:
%       color
%       mark
%       marksize
%       legendtitle
%
%%%%%%%%%%%%%%%%%%%%%%%%%%%%%%%%%%%%%%%%%%%%%%%%%%%%%%%%%%%%%%%%%%%%%%%%%%%%%%%%

\begin{document}
\begin{landscape}
\begin{center}

\large{<<:title>>}
\\
~
\\[<<:titlevspace>>]

%%%%%%%%%%%%%%%%%%%%%%%%%%%%%%%%%%%%%%%%%%%%%%%%%%%%%%%%%%%%%%%%%%%%%%%%%%%%%%%%
% AXIS 1 for D = 0
%%%%%%%%%%%%%%%%%%%%%%%%%%%%%%%%%%%%%%%%%%%%%%%%%%%%%%%%%%%%%%%%%%%%%%%%%%%%%%%%
\begin{tikzpicture}[font=<<:fontsize>>]

%%%%% M0
\begin{axis}[%
    width=<<:axiswidth>>,
    height=<<:axisheight>>,
    scale only axis,
    xmin=<<:xmin>>,
    xmax=<<:xmax>>,
    ymin=<<:mtrymin>>,
    ymax=<<:mtrymax>>,
    axis x line=none,
    hide y axis
]
\addplot[name path=zero, draw=none, forget plot] {0};

<<#:m0segments>>
\addplot[
    name path=<<pathname>>,
    color=black,
    on layer={axis foreground},
    solid,
    line width = <<~:linewidthmtr>>
]
coordinates{
    <<coordinates>>
};
\label{path-<<pathname>>}
<</:m0segments>>

\end{axis}

%%% WEIGHTS
\begin{axis}[
    width=<<:axiswidth>>,
    height=<<:axisheight>>,
    scale only axis,
    xmin=<<:xmin>>,
    xmax=<<:xmax>>,
    title={$d = 0$},
    xlabel={<<:xlabel>>},
    ymin=<<:weightymin>>,
    ymax=<<:weightymax>>,
    ylabel={<<:ylabelweights>>},
    y label style={
        at={(ticklabel* cs:<<:ylabelbuffer>>)},
        font=\small,
        anchor=west,
        align=center,
        text width=<<:ylabeltextwidth>>,
        rotate=-90
    },
    axis x line*=bottom,
    axis y line*=left,
    separate axis lines,
    legend columns=<<:legendcols>>,
    legend to name=legendsave,
    legend style={
        legend cell align=left,
        align=left,
        text opacity=1
    },
    every axis legend/.append style={nodes={right}}
]

% inherit properties associated with the `path-mtr01` path name, which is the
% path of the first segment of MTR for d = 0
\addlegendimage{/pgfplots/refstyle=path-mtr01}
\addlegendentry{<<:mtrlegendtext>>}

<<#:d0weights>>
\addplot[
    name path=<<pathname>>,
    forget plot,
    color=<<color>>,
    mark=<<mark>>,
    mark size=<<marksize>>,
    mark repeat=<<~:markrepeat>>,
    mark phase=<<~:markphase>>,
    solid,
    line width=<<~:linewidth>>
]
coordinates{
    <<coordinates>>
};
\label{path-<<pathname>>}
<</:d0weights>>

<<#:legend>>
\addlegendimage{
    solid,
    color=<<color>>,
    mark=<<mark>>,
    line width=<<~:linewidth>>,
    mark size=<<marksize>>
}
\addlegendentry{<<legendtitle>>}
<</:legend>>

\end{axis}

\end{tikzpicture}
\qquad
%%%%%%%%%%%%%%%%%%%%%%%%%%%%%%%%%%%%%%%%%%%%%%%%%%%%%%%%%%%%%%%%%%%%%%%%%%%%%%%%
% AXIS 2 FOR D = 1
%%%%%%%%%%%%%%%%%%%%%%%%%%%%%%%%%%%%%%%%%%%%%%%%%%%%%%%%%%%%%%%%%%%%%%%%%%%%%%%%
\begin{tikzpicture}[font=<<:fontsize>>]
%%%%% M1
\begin{axis}[%
    width=<<:axiswidth>>,
    height=<<:axisheight>>,
    scale only axis,
    xmin=<<:xmin>>,
    xmax=<<:xmax>>,
    ymin=<<:mtrymin>>,
    ymax=<<:mtrymax>>,
    axis x line=none,
    axis y line*=right,
    ytick={0, .25, 0.5, .75,  1},
    ylabel={<<:mtrylabel>>},
    y label style={
        at={(ticklabel* cs:<<:ylabelbuffer>>)},
        anchor=east,
        font=\small,
        rotate=-90
    }
]
\addplot[name path=zero, draw=none, forget plot] {0};

<<#:m1segments>>
\addplot[
    name path=<<pathname>>,
    color=black,
    on layer={axis foreground},
    solid,
    line width = <<~:linewidthmtr>>
]
coordinates{
    <<coordinates>>
};
\label{path-<<pathname>>}
<</:m1segments>>

\end{axis}
%%% WEIGHTS
\begin{axis}[
    width=<<:axiswidth>>,
    height=<<:axisheight>>,
    scale only axis,
    xmin=<<:xmin>>,
    xmax=<<:xmax>>,
    title={$d = 1$},
    xlabel={<<:xlabel>>},
    ymin=<<:weightymin>>,
    ymax=<<:weightymax>>,
    hide y axis,
    axis x line*=bottom,
]

<<#:d1weights>>
\addplot[
    name path=<<pathname>>,
    forget plot,
    color=<<color>>,
    mark=<<mark>>,
    mark size=<<marksize>>,
    mark repeat=<<~:markrepeat>>,
    mark phase=<<~:markphase>>,
    solid,
    line width=<<~:linewidth>>
]
coordinates{
    <<coordinates>>
};
\label{path-<<pathname>>}
<</:d1weights>>

\end{axis}

\end{tikzpicture}
\\
\ref{legendsave}

\end{center}
\end{landscape}
\end{document}
